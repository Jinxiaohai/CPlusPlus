\chapter{现成的盛宴}
\section{标准库String}
标准库类型string表示可变长的字符序列,%
使用string类型必须首先包含string头文件,%
且打开std命名空间,因为string位于std命名空间中。%
\begin{lstlisting}
  #include <string>
  using std::string;
\end{lstlisting}
\begin{table}[htpb]
  \centering
  \begin{tabular}{|l|l|}\hline
    \multicolumn{2}{|c|}{{\color{red}{初始化string对象的方式}}} \\\hline 
    stirng s1 & 默认初始化,s1是空串。  \\\hline 
    string s2(s1) & 拷贝初始化。 \\\hline 
    sting s2 = s1 & 拷贝初始化。 \\\hline 
    string s3(\textquotedblleft{value}\textquotedblright{})&s3是字面值的副本,{\color{red}{除了字面值最后的那个空字符串。}} \\\hline
    stirng s3 = \textquotedblleft{value}\textquotedblright & 等价于上面的那种初始化方法 \\\hline
    sting s4(n, \textquoteleft{c}\textquoteright{}) & 把s4初始化为由连续n个字符c组成的串。\\\hline
  \end{tabular}    
\end{table}

\subsection{string支持的操作}
\begin{table}[htpb]
  \centering
  \begin{tabular}{|l|l|}\hline
    \multicolumn{2}{|c|}{{\color{red}{string支持的操作}}} \\\hline
    os $<<$ s & 输出 \\\hline
    is $>>$ s & 读取 \\\hline
    getline(is, s) & 从is中读取一行赋给s,并返回is。\\\hline
    s.empty() & 判断是否为空 \\\hline
    s.size() & 返回s中字符的个数 \\\hline
    s[n] & 返回s中第n个字符的引用,从0开始计数。\\\hline
    s1+s2 & \\\hline
    s1=s2 & \\\hline
    s1==s2 & \\\hline
    s1!=s2 & \\\hline
    $<, <=, >, >=$ & \\\hline
  \end{tabular}
\end{table}
\subsection{读取string对象}
在执行读取操作时,string对象会自忽略开头的空白并从第一个真正的字符开始读起,%
直到遇到下一个空白为止。%
\subsection{使用getline读取一整行}
getline函数的参数是一个输入流和一个string对象,%
函数从给定的输入流中读取内容,%
直到遇到换行符为止(注意换行符也被读了进来),%
然后把所读的内容存入到那个string对象中去(注意不存在符)。%
和输入运算符一样,getline也会返回它的流参数。%
\alertinfo{出发getline函数返回的那个换行符实际上被丢掉了,%
  得到的string对象中并不包含该换行符。%
}
\subsection{string::size\_type类型}
string的size()函数返回的是一个string::size\_type类型的值,%
这是string的一个配套类型,类似下面的情状(自己歪歪的,而且明显知道自己歪歪的不对):%
\begin{lstlisting}
  class string
  {
    typedef unsigned int size_type;
  };
\end{lstlisting}
\subsection{string的加法}
两个string对象相加得到一个新的string对象,%
其内容是把左侧的运算对象与右侧的运算对象串接而成。%
\subsection{字面值和string对象相加}
当把string对象和字符字面值及字符串字面值混在一条语句中使用时,%
必须确保每个加法运算($+$)的两侧的运算对象至少有一个是string,%
这里包括加法的左结合性。
\subsection{范围for语句}
{\color{red}{如果想对string对象中的每个字符做点儿什么操作,%
    目前最好的办法就是使用}}{\bfseries{范围for语句}}。%
{\color{seagreen}{这种语句遍历给定序列中的每个元素并对序列中的每个值执行某种操作。}}
\begin{lstlisting}
  for(declaration : expression)
  {
    statement;
  }
\end{lstlisting}
其中expression表示一个{\bfseries{序列}}。%
declaration部分负责定义一个变量,%
该变量将被用于访问序列中的每个元素。%
每次迭代,declaration部分的变量会被初始化为expression部分的下一个元素值。%
一般情况下,序列中的每个原子元素会被拷贝到declaration。%
如果想要改变序列中的原子元素,则必须把循环变量定义成引用类型。%
\subsection{只改变一部分值}
要想访问string对象的单个字符有两种方式:
\begin{itemize}
\item{使用下标。}
\item{使用迭代器。}
\end{itemize}
\alertinfo{{\color{red}{下标运算符[ ]}}接收的输入参数是string::size\_type类型的值,%
  这个参数表示要访问的字符的位置。{\color{seagreen}{返回值是该位置上字符的引用。}}
}
任何表达式只要它的值是一个整型值就能作为索引。%
使用超出索引范围的下标操作将引发不可预知的结果。%
%%% Local Variables:
%%% mode: latex
%%% TeX-master: t
%%% End:
