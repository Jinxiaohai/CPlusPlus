\message{ !name(chapter05.tex)}
\message{ !name(chapter05.tex) !offset(218) }

    protected:
    std::size_t quantity = 0;
    double discount = 0.0;
  };
\end{lstlisting}
和我们之前定义的Bulk\_quote类一样,%
Disc\_quote也分别定义了一个默认构造函数和一个接受四个参数的构造函数。%
尽管我们不能定义这个类的对象,%
但是Disc\_quote的派生类构造函数将会使用Disc\_quote的构造函数来构建各个派生类对象的Disc\_quote部分。%
其中,接受四个参数的构造函数将前两个参数传递给Quote的构造函数,%
然后直接初始化自己的成员discount和quantity。%
默认构造函数则对这些成员进行默认初始化。%
\alertwarning{
  值的注意的是,我们也可以为纯虚函数提供定义,%
  不过函数体必须定义在类的外部。%
  也就是说,我们不能在类的内部为一个=0的函数提供函数体。%
}
\subsection{含有纯虚函数的类是抽象基类}
含有纯虚函数的类是{\bfseries{抽象基类}}。%
抽象基类负责定义接口,而后续的其他类可以覆盖该接口。%
我们不能创建一个抽象基类的对象。%
\alertinfo{
  我们不能创建抽象基类的对象。%
}
\subsection{派生类构造函数只初始化它的直接基类}
{\color{red}{\large{每个类各自控制其对象的初始化过程,派生类构造函数只初始化它的直接基类。}}}

\section{访问控制与继承}
每个类分别控制自己的成员初始化过程。%
与之类似,每个类还控制着其成员对于派生类来说是否可访问。%
\subsection{受保护的成员}

\message{ !name(chapter05.tex) !offset(-36) }
