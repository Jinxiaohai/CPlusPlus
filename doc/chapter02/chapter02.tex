\chapter{基础知识}
\section{简单语句}
表达式语句的作用是执行表达式并丢弃掉求值结果。%
\subsection{复合语句}
复合语句是指花括号括起来的语句和声明的序列,%
复合语句也被称为块。一个块就是一个作用域,%
在块中引入的名字只能在块内部以及嵌套在块内的子块中进行使用。%
通常,名字在有限的作用域中可见,该区域从名字定义处开始,%
到名字所在的块的结束为止。%
\subsection{语句作用域}
可以在if、switch、while和for语句的控制结构内定义变量。%
定义在控制结构当中的变量只在相应语句的内部可见,%
一旦语句结束,变量也就超出其作用域了。%
\section{条件语句}
C++提供了两种条件执行的语句。%
一种是if语句,它根据条件决定控制流。%
另一种是switch语句,它计算一个整型表达式的值,%
然后根据这个值从几条执行路径中选择一条。
\subsection{if语句}
if语句的作用是:判断一个指定的条件是否为真,根据判断结果决定是否执行另外一条语句。%
if语句包含两种形式:一种包含else分支,另外一种没有。%
常见的语法形式为:
\begin{lstlisting}
  if(condition)
  {
    statement
  }
\end{lstlisting}
{\bfseries{if else语句的形式是}}
\begin{lstlisting}
  if(condition)
  {
    statement;
  }
  else
  {
    statement2;
  }
\end{lstlisting}
条件检测部分必须用括号括起来。
%%% Local Variables:
%%% mode: latex
%%% TeX-master: t
%%% End:
